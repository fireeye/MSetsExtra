\documentclass[a4paper,10pt,oneside]{article}

\usepackage[utf8]{inputenc}
\usepackage[a4paper]{geometry}
\usepackage{hyperref}
\usepackage{url}
\usepackage{color}
\usepackage{natbib}
%\usepackage{todonotes}

\newcommand{\comment}[1]{\todo{\textcolor{red}{#1}}}
\newcommand{\commentTT}[1]{\todo{\textcolor{blue}{TT: #1}}}
\newcommand{\commentHT}[1]{\todo{\textcolor{green}{HT: #1}}}
\newcommand{\commentJM}[1]{\todo{\textcolor{yellow}{JM: #1}}}

\newcommand{\code}[1]{\texttt{#1}}

\title{Efficiently Executable Sets used by FireEye}

\author{FireEye Formal Methods Team Dresden%
  % respect EU directive 2003/58/EC ;-
  \footnote{\url{formal-methods@FireEye.com}, FireEye Technologie Deutschland GmbH, Wilsdruffer Strasse 27, 01067 Dresden, Germany, Amtsgericht Dresden HRB 33246, Geschäftsführer: Alexa King, Mark Hegarty}}
\date{1st July 2016}

\begin{document}
\maketitle

\section{Introduction}



FireEye provides cybersecurity solutions for detecting, preventing and
resolving cyber-attacks. In
order to increase the confidence in many of our products we perform proofs of security-critical software
components in parallel with development of the components themselves.
We
carry out the proofs in a Coq~\cite{coq} model that follows the structure of the
implementation, but is much more abstract. The resulting gap is
bridged by model-based testing.

Our testing framework is developed in OCaml and hinges on Coq's code
extraction to obtain an executable version of the model. We execute
millions of test cases, each with a state comprising thousands of
objects. Thus, performance is a critical aspect: we need to fine-tune
the used data structures based on the frequency of the operations
performed.

Coq's sets library, while well-suited for mathematical reasoning,
proved to be a bottleneck for our testing framework.  In this note, we
describe an extension of this library that drastically improves its
usefulness in our development.

\section{Fine Tuned Set Implementations for Code Extraction}

Coq (as of version 8.4pl6) provides the module \code{MSetInterface}
which contains interfaces for sets. There are implementations using
sorted and unsorted lists (both without duplicates) as well as AVL and
RBT trees for these interfaces. While these
implementations---particularly the ones based on binary trees---
extract to reasonably efficient code, we nevertheless were struggling with performance
issues. Profiling revealed several problems related to the extracted
set code: large sets of integers use a lot of space; slow insert and
union operations are a huge issue for one of our algorithms and a slow
specialised filter operation for another one. In the following we
explain how we solved each of these challenges by providing general
purpose extensions to Coq's sets library as well as special purpose
set implementations.


\paragraph{Interval Sets}
In our setting, we need large sets of integers. They contain either
only very few integers or nearly all the integers between 0 and
65535. The AVL
tree set implementation was time efficient, but used large amounts of
memory and, when marshalled, disc-space.  By providing a set
instantiation\footnote{module \code{MSetIntervals}} that uses lists of
intervals internally, we could shrink the memory and disc-space
requirements drastically.


\paragraph{Lists with Duplicates}
Some of our algorithms collect a set of results by adding single
results and combining result sets. They guarantee that results are not
added multiple times to a set. For such requirements, lists are an
efficient, commonly used datastructure. However, lists don't provide a
high-level set view. On the other hand, all standard Coq set
implementations are inefficient for our use case, because they perform
an expensive membership test when inserting elements.

We provide an implementation of the set interfaces using lists with
duplicates\footnote{module \code{MSetListWithDups}} and we allow
fold to visit the same element multiple times.  To achieve
this, we removed the requirement that the element lists contain no
duplicates from Coq's set interface \code{WSetsOn}\footnote{see new
  interface \code{WSetsOnWithDups}}.


\paragraph{Fold With Abort}
Folding over all elements of a set is a core operation, which has
close ties to the concept of iterators in languages like C++. However,
while iterators allow early abort, this is not possible with
the standard fold operation. As a result, many efficient iterator
algorithms become inefficient when implemented via folding.

Fold with abort operations (see e.\,g.~\cite{lammich10itp}) are an
answer to this issue.  We provide interfaces for a family of such
\emph{fold with abort} operations and use these operations to define
very efficient filter and existence check operations. The interfaces
are instantiated for all of Coq's set implementations.

In our application, we use large sets of intervals implemented via AVL
trees. Using fold with abort operations, we can very efficiently find
all intervals overlapping with a given interval.


\section{Conclusion}

One of our typical test cases used to spend about 90 s in Coq
extracted code (and 25 s in extra code).  By using lists with
duplicates, the runtime of the Coq extracted code could be reduced to
about 9 s. Using interval sets reduced the overall disc-space required
to store a test result to just about a quarter (while marginally
improving the runtime).  Finally, using fold with abort operations
reduced the needed runtime to less than a second for the Coq extracted
code. We believe that the set implementations used to achieve these
improvements are useful in general and therefore provide them to the
community~\cite{fireeye-coq-sets}.

The issues described here for the set interfaces are an instance of a more
general issue we have encountered at FireEye. Many of Coq's libraries
are aimed at theorem proving and provide good mathematical
abstractions. However, they fall short of providing efficiently
executable extracted code. We believe that this is an important
deficiency of the Coq library and we would be delighted if the work
presented here calls attention to this problem and becomes a first
step towards providing a comprehensive library fine-tuned for code
extraction.


\bibliographystyle{plain}
\bibliography{mset} 



\end{document}

%%% Local Variables:
%%% mode: latex
%%% TeX-master: t
%%% End:
